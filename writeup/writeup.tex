\documentclass[a4paper]{article}

\usepackage[pages=all, color=black, position={current page.south}, placement=bottom, scale=1, opacity=1, vshift=5mm]{background}
\SetBgContents{
	\tt Adam Zawieruhca
}      

\usepackage[margin=1in]{geometry} % full-width

% AMS Packages
\usepackage{amsmath}
\usepackage{amsthm}
\usepackage{amssymb}

% Unicode
\usepackage[utf8]{inputenc}
\usepackage{hyperref}
\hypersetup{
	unicode,
%	colorlinks,
%	breaklinks,
%	urlcolor=cyan, 
%	linkcolor=blue, 
	pdfauthor={Author One, Author Two, Author Three},
	pdftitle={A simple article template},
	pdfsubject={A simple article template},
	pdfkeywords={article, template, simple},
	pdfproducer={LaTeX},
	pdfcreator={pdflatex}
}

% Vietnamese
%\usepackage{vntex}

% Natbib
\usepackage[sort&compress,numbers,square]{natbib}
\bibliographystyle{mplainnat}

% Theorem, Lemma, etc
\theoremstyle{plain}
\newtheorem{theorem}{Theorem}
\newtheorem{corollary}[theorem]{Corollary}
\newtheorem{lemma}[theorem]{Lemma}
\newtheorem{claim}{Claim}[theorem]
\newtheorem{axiom}[theorem]{Axiom}
\newtheorem{conjecture}[theorem]{Conjecture}
\newtheorem{fact}[theorem]{Fact}
\newtheorem{hypothesis}[theorem]{Hypothesis}
\newtheorem{assumption}[theorem]{Assumption}
\newtheorem{proposition}[theorem]{Proposition}
\newtheorem{criterion}[theorem]{Criterion}
\theoremstyle{definition}
\newtheorem{definition}[theorem]{Definition}
\newtheorem{example}[theorem]{Example}
\newtheorem{remark}[theorem]{Remark}
\newtheorem{problem}[theorem]{Problem}
\newtheorem{principle}[theorem]{Principle}

\usepackage{graphicx, color}
\graphicspath{{fig/}}

%\usepackage[linesnumbered,ruled,vlined,commentsnumbered]{algorithm2e} % use algorithm2e for typesetting algorithms
\usepackage{algorithm, algpseudocode} % use algorithm and algorithmicx for typesetting algorithms
\usepackage{mathrsfs} % for \mathscr command

\usepackage{lipsum}

% Author info
\title{Minimizing Page Faults on Bloom Filters
\author{Adam Zawierucha (adz2)}

\date{
	Rice University \\ zawie@rice.edu}%
%	\today
}


\begin{document}
	\maketitle
	
	\begin{abstract}
		Bloom filters \cite{Bloom} are used ubiquitously due to their speed and memory efficiency in theory and in practice.
However, the standard implementation of sufficiently large bloom filters suffers from page faults.
In this paper we propose a bloom filter implementation that guarantees one page access per operation. 
This minimizes page faults, thereby drasticaly improving efficiency.
We will show theoretically and empirically that our hierarchical implementation is expected to be faster than the standard implementation without alterating the false positive rate.

\noindent\textbf{Keywords:} probabilistic data structures, bloom filters, memory hierarchy, implementation, page fault analysis
	\end{abstract}
	

	\section{Introduction}
	Bloom filters are space efficient probabilistic data structures that implement set operations developed by Bloom in 1970 \cite{Bloom}.
The trade-off of the afformentioned space efficiency is a probabilistic response.
When an element is queried for membership, a ``false'' response means 
the element is definitely not a member, but a ``true'' response only means
the element \textit{may} be a member. 
The rate at which the data structure incorrectly reports that an element has been inserted is called the \textit{False Positive Rate (FPR)}.
This probabilistic feature allows Bloom filters to be extraordinarly space efficient,
only needing a constant number of bits per stored element irrespective of the elements size.

Due to Bloom filter's space efficiency they have been adopted in many domains from network security \cite{GERAVAND20134047} to bioinformatics \cite{btu558}.
Any improvement in performance (without negatively alterating the false positive rate) could lead to reduce latency in network calls or speed-ups in sequenceing DNA.
This motivates our proposed solution which exploits computer systems' memory hierarchy to provide better performance without increasing the false positive rate of bloom filters.
s
	\section{Literature Survey}
	% In my literature survey, I discovered a similar approach except making bloom filters hierarchial.
% Tim Kaler's proposed cache-efficient bloom filters \cite{Kaler} makes sub-bloom filters of size cache-size; thus, their idea is very similar to mine except they do it on a smaller unit of memory.

% Evgeni Krimer and Mattan Erez used a power-of-two choice principle within blocked-bloom filters to decrease the false positive rate.
% Instead of simply selecting one block to write into, they choose multiple.
% \cite{Krimer}.


\color{blue}
\textbf{Note:} I decided to do a literature survey of different data structures solving this problem \textit{after} I explored this idea. 
I wanted to have the joy of discovering whether or not my idea would work on my own!
\color{black}
Felix Putze et al. developed bloom filters with better cache efficiency and requiring less hash bits than the standard bloom filter \cite{Putze}.
One of their implementations is similar to mine. They discretize a bloom filter into \textit{cache size} (64 bytes) chunks instead of \textit{page size} chunks (4096 bytes).
This leads to even \textit{faster} improvements at the cost of a higher false positive rate. Remember, the smaller you make the sub-bloom filter, the worse the Euler-identity based approximation becomes.
They also recompute random bit patterns: instead of setting $k$ bits in the bloom filter, they hash once to create a mask that sets multiplem bits \cite{Putze}. 
This saves them time while performing hash functions. The combination of these techniques gave them much better performance over the standard implementation with slight false positive rate increases.
I would be interested in further exploring if some pattern masking can be applied to our bloom filter. This may require another level in our hirarchy to give us small filters that we can efficiently mask over.

Rafael P. Laufer et al. developed a ``tamper proof'' Generalized Bloom Filter \cite{Laufer2007GeneralizedBF} to counteract an inherent vulnerability in bloom filters.
The authors call the exploitation of this vulnerability an ``all one'' attack, where a malicious agent floods the bloom filter with requests to set all the bits to one, artificially increasing the false positive rate.
This may slow down the program that the bloom filter is employed in and marks elements as members of the set at a higher rate than it should.
While this paper does not seek to solve a similar problem to our implementaiton, it is important to note that the hierarchy we imposed on our bloom filter
may make our implementation more susecptible to this kind of attack. If a malicious agent wanted to cause a high false positive rate for a certain kind of element, they need not find a system to set all the bits.
Instead, they would only have to exploit one hash function: the hash function that selects bloom filters. 
This may pose a security risk.
Thus, for production uses, it may be wise to use a cryptographically secure hashing algorithm for this ``choke point'' hash function.
However, we could also employ technieques discussed by Laufer, such as having multiple hash functions that set and reset bits accross the bit vector \cite{Laufer2007GeneralizedBF}.
In any case, further research should be done to prevent our implementation from being exploited.

	\section{New Implementation Proposal}
	\newcommand{\sep}{\hspace*{0.15in}}
\newcommand\cell{%%
    \fbox{\rule{0.15in}{0pt}\rule[-0.5ex]{0pt}{0.15in}}}

Before we discuss the proposed solution, let us highlight the weakness of the standard implementation.
The standard implementation allocates a bit vector of size $m$ bits. Typically, this is set to be $\times 10$ the expected number of elements it will hold.
The figure below represents the bit vector of length $m$.

\begin{center}
    \textit{Standard Bloom Filter}
    \vspace{10pt}\\
    $\text{Bit Vector: } \cell\cell\cell\cell\cell\cell\cell\cell\cell\cell\cell\cell \ldots$
    \vspace{10pt}\\
    \textbf{Figure 1}
\end{center}

Notice, that if $m$ is sufficiently large, it will span accross multiple pages of memory. 
Thus, when we read and write to the underlying bit vector, we may page fault for every bit set, slowing down our insertions or queries.

Our proposal is instead to allocate $w$ bit vectors of size $P$ bits, where $P$ is the number of bits of the computer system's page size, and $w$ is a number chosen such that $m = w \cdot P$.
Notice, our proposed implementation uses the same amount of memory, but conceptually splits the bit vector into page size chunks.

\begin{center}
    \textit{Hierarchical Bloom Filter}
    \vspace{10pt}\\
    $\text{Bit Vector 1: }\cell\cell\cell\cell \sep \text{Bit Vector 2: }\cell\cell\cell\cell \sep \ldots \sep \text{Bit Vector $w$: }\cell\cell\cell\cell$
    \vspace{10pt}\\
    \textbf{Figure 2}
\end{center}


	\section{Performance Theoretical Justification}
	In this section we will justify why the hierarchical implementation will outperform the standard implementation in theory while preserving effectiveness.
First, we will \textbf{calculate expected number of page faults} per implementation.
Second, we will \textbf{find the theoeretical false positive rate}, which will guide us in our parameter selection for the implementation.

\subsection{Expected Page Faults}
Page faults occur when the operating system must fetch memory from a source higher in the memory hierarchy to be used by the process.
Whenever a page fault occurs, the program must be halted unneccesarily to resolve the page fault, which could require relatively slow I/O operations such as checking the TLB or loading the page from memory or disk.
This leads to slower performance.

It is next to impossible to know when accessing a page will cause a page fault as this is highly dependant on the operating system.
In general though, the more memory you are using, the higher the likelyhood a pagefault will occur.
In this analysis we will assume more page accesses is coorelated with a higher liklihood of page faulting.
We can formly compute the expected number of different pages that will be asccessed (unlike whether or not it will page fault).

Note, when discussing page faults in this section, we will only discuss the page faults caused due to accessing the underlying bit vector of the bloom filter.
Natuarelly, page faults can occur while running the underlying code of the bloom filter or running the hash functions, but this should be rare and would realistically only cause one page fault.

First, we will discussed the expected number of page faults for the standard implementation.

Let $A$ be a random variable representing the number of pages accessed. 
Let $P$ be the number of bits in a page and let $m$ be the number of bits in the underlying bitvector.
Suppose there are $k$ hash functions.
We now define the indictor variable $A_i$ which is $1$ if bit $i$ is set, $0$ otherwise.
$$A = A_1 + A_2 + \ldots + A_{m/P}$$
Thus, the expected number of pages accessed can be found by computing the expected count that any page is accessed by the linearity of expectation:
$$E(A) = \sum_{i=1}^{m/P} E(A_i) = \frac{m}{P} \cdot E(A_i) \text{ for arbitrary $i$}$$
The probability that $A_i$ is accesed at least once is the inverse of it being never accessed.
Assuming that our hash function is uniform, we expect it to pick any particular page with probability $\frac{1}{m/P} = \frac{P}{m}$.
Thus, the probablity $A_i$ is accessed at least once is:
$$E(A_i)  = 1 - (1 - \frac{P}{m})^k$$
Ergo, we have a closed form equation for the expected number of page faults for a given operation:
$$\text{Expected number of page accesses (Standard)} = \frac{m}{P} (1 - (1 - \frac{P}{m})^k)$$

We will use this formula to compute the expected number of page faults for two reasonable cases.
First, suppose you wanted a bloom filter to store $32,768$ elements with an underlying bit vector of size $\times 10$ that.
Note, most computer systems have a page size of $4096$ bytes, so this works out to require exactly $10$ pages of memory.
Additionally, suppose we pick the optimal bloom filter parameter and set $k=7$.
Under this scenario, we anticpate:
$$\text{Expected \# accesses (Standard)} = 10(1-0.9^7) \approx 5.2$$
If we wanted to store $327,680$ elements under a similar setting, then the number of pages faults would be:
$$\text{Expected \# accesses (Standard)} = 100(1-0.99^7) \approx 6.8$$

As we can see, even using a relatively small bloom filter, we anticpate almost every bit to be located in an entirely different page.
This means we can page fault multiple times during a single operation!

Since our operation limits bit setting and reading to a single page per insertion or query, we need to access exactly one page!
$$\text{Number of accesses (Hierarchical)} = 1$$

Thus, our proposed hierarchical solution limits the number of page faults to at most one!
Therefore, we anticpate much better performance.
\subsection{False Positive Rate}

\begin{equation}
    f_p = (1 - (1 - \frac{1}{m}^{nk}))^k
\end{equation}




\subsubsection{Hierarchical Implementation}

Suppose we have a allocated $m$ bits in total chunked into $w$ bit vectors of size $P$ bits:
$$ m = wP$$
Moreober, suppose we anticipate to insert $n$ elements are we allocate


Assuming our hash function's output is uniform, we can conclude that for
\begin{equation}
    f_p = (1 - (1 - \frac{1}{m}^{nk}))^k
\end{equation}

	\section{Experiments}
	We will now emperically validate that the hierarchical implementation is more time efficient than the standard implementation without sacrificying accuracy.
Two experiments will be conducted.
First, we will \textbf{measure elapsed time as insertions scale} of each implementation.
We anticpate that the hierarchical implementation will take less time than the standard implementation for any sufficiently large value of insertions.
Second, we will \textbf{measure false positives as we scale bits allocated per element} of each implementation.
We anticipate that the hierarchial and standard implementation will be approximately same as the theoerical false positive rate discussed before.

For both of these sections, we pseudorandomly generate keys to both insert and query.
This is done by randomly selecting 94 bytes some amount of times (this is varier per experiment) using C's \texttt{rand} function.
This is sufficiently random for our experimental purposes. 
%%%%%%%%%%%%%%%%%%%%%%%%%%%%%%%%%%%%%%%%%%
%%%%%%%%%%%%%%%%%%%%%%%%%%%%%%%%%%%%%%%%%%
%%%%%%%%%%%%%%%%%%%%%%%%%%%%%%%%%%%%%%%%%%

\subsection{Comparing Time Efficiency}

For this experiment, we seek to validate that the hierarchical implementation performs better than the standard implementation as the number of insertions grow.
\textbf{Hypothesis: The hierarchical bloom filter will take less time than the standard implementation to insert $n$ keys for any $n$.}

\subsubsection{Experimental Settings}

The experiment will run as follows. For each implementaiton run the following procedure:
\begin{enumerate}
    \item Generate $n$ random keys of length $8$.
    \item Generate a bloom filter of both varianets of size $10n$. Use the optimal theoeretical configuraiton for each bloom filter (i.e, $k=7$, $l=1$).
    \item Time how long it takes to insert all $n$ keys into each of the bloom filters. Report this number.
    \item Repeat for various sizes of $n$.
\end{enumerate}
Repeat this entire process $3$ times.

\subsubsection{Results}
\begin{center}
    \includegraphics[width=13cm]{scale-nm.png}
\end{center}
The slope of the line of best fits that are plotted are as follows whre $n$ is millions of insertions:
\begin{itemize}
    \item The best fit line for the standard implementation is: $t = (1.1078 \pm 0.01446 )\cdot n - (5.78611   \pm 0.7407)$
    \item The best fit line for the hierarchical implementation is: $t = (0.893221 \pm 0.01166 )\cdot n - (4.52759 \pm 0.597 )$
\end{itemize}
We will disregard the constant as we care about how these data structures perform as input volume scales. We can compute the efficiency difference by dividing the slopes:
$$\text{Hierarchical Implementation Slope}/ \text{Standard Implementation Slope} = \text{Efficiency Difference}$$
$$(0.893221 \pm 0.01166 \text{ seconds/operation}) / (1.1078 \pm 0.01446 \text{ seconds/operation}) = 0.806301679 \approx 80\%$$
In other words, our implementation takes $80\%$ less time per operation to complete. 
Thus, for any time frame, if the standard implementation performs one operation, the hierarchical implementation is expected to perform $1/80\% = 1.25$ operations.
Thus, our experiment shows that the hierarchical implementation performs $25\%$ more operations per second than the standard implementation!

%%%%%%%%%%%%%%%%%%%%%%%%%%%%%%%%%%%%%%%%%%
%%%%%%%%%%%%%%%%%%%%%%%%%%%%%%%%%%%%%%%%%%
%%%%%%%%%%%%%%%%%%%%%%%%%%%%%%%%%%%%%%%%%%

\subsection{Comparing False Positive Rate}

For this experiment, we seek to validate that the hierarchical implementation does not have a worse false positive rate than the standard implementation.
\textbf{Hypothesis: We expect them to have approximately the same false positive rate as the theoeretical expectation discussed earlier.}

\subsubsection{Experimental Settings}

The experiment will run as follows. 
\begin{enumerate}
\item Generate $150,000$ random ``insertion'' keys (of length 16).
\item Generate $150,000$ random ``false'' keys to query distinct from the insertion keys (of length 15).
\item For each implementaiton run the following procedure:
\begin{enumerate}
    \item Let $BPE$ be the bits per element (e.g $BPE = 1$ or $BPE = 10$).
    \item Generate a bloom filter of size $150,000\cdot BPE$.
    \item Insert all the `insertion'' keys and query all the ``false'' keys and measure how many of them the bloom filter return as being a member. Report this number.
    \item Repeat for various values of $BPE$.
\end{enumerate}
\item Repeat this procedure again $3$ times with different insertion and false keys.
\end{enumerate}

\subsubsection{Results}
\begin{center}
    \includegraphics[width=14cm]{fp.png}
\end{center}

As we can see, both the standard and hierarchical implementation closely match the theoeretical expectation. 
Both implementations do worse than theoeretically expected if bloom filters are overpacked, but after Bits per Elements is greater than 6, both implementations are very close to theoeretical expectation ($\pm 0.0005$) or better.

\subsection{Conclusion}
Our experiments have supported both of our theoretically justified hypothesises.
We have demonstrated that the hierarchical implementation is more efficient than the standard implementation; it can perform $25\%$ more operations per second!
Additionally, we have verified our that our implementation is just as effective as the standard implementation. 



	% I run three experiments on my data structure.

	% \subsection{Measure False Positive Rate}
	
	% For this experiment, we seek to measure how the false positive rate of our Hierarchical bloom filter compares to the standard bloom filter. 
	% Our theoeritcal anaylsis shows that these should be equivalent.
	
	% \textbf{Hypothesis:} \textit{The hierarchial bloom filter will have an identical false positive rate.}

	% \subsubsection{Experimental Settings}
	% The experiment will run as follows. First, generate $N$ random keys to insert and $N$ random keys to query by that are all distinct from the insertion keys.
	% We choose $N= 2,000,000$. Then, for each bloom filter variant, generate bloom filters of various sizes: a bloom filter with $2N$ bits, $3N$ bits, $\ldots$ to $15N$ bits. 
	% Generate the bloom filter with the parameters that theoertically minimize false positive rate; i.e $k=7$, $l=1$.
	% Then, for each sized bloom filter, insert our $N$ keys for insertion and query $N$ false keys and compute what precentage of them are falsely accepted.
	% This measurement is the false positive rate of the variant for a certain bits per element measurement.
	% Repeat this experiment $20$ times and take the average. (The plots I generated in results only do it once, for now.)
	% This gives us a good gauge of how false positive rate behaves for each variant as we provide more memory.
	
	% Ideally, we want to see the Hierarchical bloom filter having the same false positive rate as the standard bloom filter.

	% \subsubsection{Results}
	% \begin{center}
	% 	% \includegraphics[width=10cm]{../plots/fp.png}
	% \end{center}

	% We can see interesting results here. 
	% The hierarchial bloom filter has substantially worse false positive rates sometimes. 
	% You can see sometimes the false positive bad becomes extraordinarily bad. 
	% These upticks are likely due to bad selections of seed for the choke-point hash function. Thus, this leads to a lot of collisions and therefore a high false positive rate.

	% For the filnal paper I will run the experiment multiple times and mean to get a more meaningful result.

	% Nonetheless, this demonstrates a flaw in my proposed data structure which I will discuss more in my final paper.
	% It is interesting that the theoeretical analysis predicts that they will have the same bound, but in practice we see worse false positive rate.
	% \subsection{Measure Throughput for small $m$}
	
	% For this experiment, we seek to validate two things 
	% \begin{enumerate}
	% 	\item The run time is constant with respect to the size of the bloom filter
	% 	\item The hierarchical bloom filter has similar speeds to the standard bloom filter for low page sizes.
	% \end{enumerate}

	% \textbf{Hypothesis:} \textit{The hierarchial bloom filter will work the same 1 page size, and slowly become better but platue when the page count becomes 7.}
	% \subsubsection{Experimental Settings}
	% The experiment will run as follows.
	% First, generate $N=10,000,000$ random keys. 

	% Then, we will run the following experiment on the two variants and plot the results.
	% Use the optimal theoeretical configuraiton for each bloom filter (i.e, $k=7$, $l=1$)
 	% \begin{enumerate}
	% 	\item Generate a bloom filter of size $m$ pages.
	% 	\item Insert all $N$ keys and time how long it takes to do the operation; we will denote the elapsed time as $t$
	% 	\item Compute and plot the throughput: $N/t$
	% \end{enumerate}
	% Repeat the experiment for $m = 1,2,\ldots,8,9,10,15,20,25$

	% \subsubsection{Results}
	% \begin{center}
	% 	% \includegraphics[width=10cm]{../plots/scale-m.png}
	% \end{center}

	% These results are positive: the hierarchial bloom filter has higher throughput than the standard bloom filter.

	% However, I have unexplained results as my bloom filter should behave identically when $m=1$. This will be explored for the final paper.
	% It is possible I miss computed or made a coding error in one or both of the implementations. This requires more digging...
	% \subsection{Measure Duration as $n$ scales}

	% For this experiment, we seek to validate that our implementation performs better than the standard implementation as $n$ grows.
	
	% \textbf{Hypothesis:} \textit{The hierarchial bloom filter will take less time to insert $n$ keys}

	% \subsubsection{Experimental Settings}
	
	% The experiment will run as follows:
 	% \begin{enumerate}
	% 	\item Generate $n = 500,000$ keys
	% 	\item Generate a bloom filter of both varianets of size $10n$. Use the optimal theoeretical configuraiton for each bloom filter (i.e, $k=7$, $l=1$).
	% 	\item Time how long it takes to insert all $n$ keys into each of the bloom filters.
	% 	\item Plot the time and results.
	% 	\item Repeat for $n = 1e6, 2e6, \ldots 120e6$
	% \end{enumerate}

	% \subsubsection{Results}
	% \begin{center}
	% 	% \includegraphics[width=10cm]{../plots/scale-nm.png}
	% \end{center}

	% These results are positive: the hierarchial bloom filter takes noticably less time to insert $n$ keys for every choice of $n$.
	% I will do more numerical analysis for the final paper. But these results are promising!
	\newpage
	\bibliography{refs}

	
\end{document}