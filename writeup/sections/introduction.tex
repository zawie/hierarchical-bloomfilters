Bloom filters are extremely space efficient probabilistic data structures developed to implement set operations developed by Bloom in 1970 \cite{Bloom}.
When an element is queried for membership, a ``false'' result means 
the element has definitely not been added, but a ``true'' only means
the element \textit{may} have been inserted. 
The rate at which the data structure incorrectly reports that an element has been inserted is called the \textit{False Positive Rate (FPR)}.
This probabilistic feature allows Bloom filters to be extraordinarly space efficient,
only needing a constant number of bits per stored element (irrespective of the elements size).

Due to Bloom filter's space efficiency they have been adopted in many domains from network security \cite{GERAVAND20134047} to bioinformatics \cite{btu558}.
Any improvement in performance (without negatively alterating the false positive rate) could lead to reduce latency in network calls or speed-ups in sequenceing DNA.
This motivates our proposed solution which exploits computer systems' memory hierarchy to provide better improvement without increasing the false positive rate of bloom filters.